\documentclass{article}

\title{Is it right to design intelligent devices with human features?}
\author{Jacob Braidley}
\date{}

\usepackage{geometry}
\geometry{margin=3cm}

\begin{document}

\maketitle

\vspace{0.5cm}

\section{Introduction}
Living in the modern era of technology, the public have become more accustomed to accepting new and innovative ideas surrounding the digital world. With significant development during the last decade, many more intelligent devices with human features have been released onto the consumer market. For Intelligent Personal Assistants (IPAs), the global market has been predicted to grow from 2016 to 2024 at an annual rate of 32.8\% \cite{tmr2016market}. These devices are responsible for assisting people in everyday life and have been designed to replicate human features in different ways. In the world today, it is not unlikely to come across intelligent devices that resemble a human in a certain way; interactions between these devices and humans concern the entirety of our daily lives \cite{securityTech2022interactions}. However, this essay will evaluate whether these devices are beneficial to society and should continue being designed, or if they can cause a negative effect on consumers and the public.

\section{The Problem}
Like all technology in development, human-like devices have been designed to solve certain issues that occur within the world. It is a large area of development and research that includes a multitude of unique devices, embodying different human features, that could benefit society in different ways.\par
Robotics has seen great advancements in recent years leading to humanoid robots that share many characteristics with humans, such as bipedal movement and facial expressions. Social robotics focuses on the aim of designing such devices that make people’s everyday lives easier through assistance \cite{wykowska2014social}, benefiting the public.\par
Smart devices are another example of intelligent devices with human features that are prevalent in society. These devices are designed to have a human voice that responds to the user after receiving oral commands, allowing for a hands-off approach when operating such device. The purpose of this technology is to provide easier methods for completing trivial daily tasks, such as playing music or finding information online.\par
As discussed, there are many intelligent devices that have been designed to have human-like features for different reasons, but regardless of the rationale there is always the same debate as to whether it is right to design such technology.

\section{In Favour}
The more likely a person is to use technology is decided upon the trust that they feel towards the device. This is even more of a factor for intelligent devices that are responsible for many things in our lives, clearly exposed to both physical and virtual personal data of the users.\par
Anthropomorphism is defined as the “tendency to attribute human characteristics to objects” \cite{woods2004anthropomorphism} and this is done to intelligent devices when they possess human features. It is believed that the more human features an intelligent device shows, the more easily anthropomorphism is achieved \cite{vanPinxteren2019features}. For example, a person is more likely to associate the capability of feelings to a robot that looks and moves like a human rather than one that does not. Furthermore, the trust that humans feel for devices is driven significantly by anthropomorphism \cite{woods2004anthropomorphism} and how well it can be applied to a certain technology. This is particularly important for social robotics where humans need to feel comfortable with the device in order for assistance such as caregiving.\par
Trust is a main factor as to whether people will purchase intelligent devices for their home. Amazon’s Echo series of devices are a set of voice command smart devices that operate from oral commands, replying to the user in a human-like voice. Such a voice has gone through development so that it can portray artificial emotions. It is thought that devices believed to show emotions, even if they are artificial, allow for better interaction between user and device \cite{buiu2011emotions}; the more trustworthy people are with this technology, the more sales are made. By 2024, the IPA market is predicted to reach \$7.9 billion \cite{tmr2016market}. The increase in quality of human-like features in intelligent devices has not only made the public more comfortable with such technology, but also increased the market value.\par
Designing intelligent devices to reflect human features is important for the technology where trust is key for the use. When a device is designed for a purpose that concerns interacting with people, it is shown that the more human-like its design, the better accepted it is by the public.

\section{Against}
Although being designed to solve social problems, some experts have indicated risks that come with giving intelligent devices human-like appearances \cite{phillips2018risks}. For example, anthropomorphism could be associated to a humanoid robot, relating people to this device which in turn increases trust. However, if this humanoid robot does not now function as expected, it can cause distrust from people \cite{phillips2018risks}. This suggests that intelligent devices with a higher likeness to humans are more likely to become unaccepted by the public. To reinforce this idea, it is key to look at the uncanny valley hypothesis constructed by roboticist Masahiro Mori \cite{mara2022uncanny, mori1970valley}. In this hypothesis, it was proposed that increasing in human likeness for devices which exhibit slight anthropomorphism results in increasing human trust, however there is a point at which the device has become too human-like, causing the feeling of eeriness and distrust by people \cite{mara2022uncanny, mori1970valley}. It is clear that intelligent devices with too much resemblance to humans can be discomforting and unlikable by the public.\par
With the increase in intelligent devices in the home, security has become a larger factor when discussing the morality of designing such technologies. Intelligent devices, in particular social robots with their human-like mobility, can pose implications on both physical and informational privacy \cite{lutz2019privacy}. These smart devices are present in people’s homes and experience their everyday lives. This exposes them to private data that a person may not want to share. In addition, for social robots with independent mobility, there poses the risk of physical privacy. With human-like intelligent devices being integrated into more parts of our social life, there is an increase in the need to think about legal consequences \cite{khoury2017legal}. If an intelligent device crosses the boundary of privacy and security then legal action must be taken, but the question is to who should be legally responsible? \cite{khoury2017legal} This issue raised makes the discussion of human-like devices difficult since there is not one correct answer to legal responsibility.\par
There is not a clear solution to the negative impacts of intelligent devices that show human features, but nevertheless it shows that there can be flaws in the implementation of this technology, and that it is more a social issue.


\section{Conclusion}
It is evident from the uncanny valley hypothesis \cite{mori1970valley}, security, and legal issues that intelligent devices with human features are not always designed correctly, leading to problems with how they are perceived by the public in terms of concern and distrust. However, when designed in the right way, these devices can be beneficial to society through helping people and providing ease to life. In conclusion, as long as they are designed well, it is right to design intelligent devices with human features.

\vspace{.5cm}
\noindent \textbf{Word count:} XXXX. (please replace XXXX by the word count).

\bibliographystyle{acm}
\bibliography{main}

\end{document}